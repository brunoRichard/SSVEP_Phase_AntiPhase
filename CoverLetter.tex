\documentclass[12pt]{article}
\usepackage{graphicx}
\usepackage{geometry}
\geometry{margin=1in}
\setlength{\parskip}{1em}   % vertical space between paragraphs
\setlength{\parindent}{0pt} % no indentation
\begin{document}

% Letterhead block
\noindent
\begin{minipage}[t]{0.6\textwidth}
% left side empty for now (can add sender info, date, etc.)
\end{minipage}%
\hfill
\begin{minipage}[t]{0.5\textwidth}
    \raggedleft
    \includegraphics[width=0.9\textwidth]{Figures/RU_H_1LINE_RED_BLACK_RGB.png}\\[1em] % Rutgers logo
    {\small
    Department of Math and Computer Sciences\\
    Rutgers University – Newark\\
    101 Warren Street\\
    Newark, NJ 07102
    }
\end{minipage}

\vspace{2cm} % space before letter body

Dear Sir/Madam,

We are pleased to submit our manuscript entitled \textit{Neural responses to binocular in-phase and anti-phase stimuli} for consideration for publication in Vision Research. This study investigates the effects of stimulus spatial and temporal phase on observer SSVEPs and explores the necessary computational components to explain our data. We develop a hierarchy of models, following the two-stage contrast gain control model (Meese et al., 2006) framework to explain the neural results.

Steady-State Visually Evoked Potentials (SSVEPs) were recorded from 15 observers in response to monocular and binocular stimulation at 3Hz, using either on-off or counterphase flicker. Across the eyes, binocular stimuli could be (i) in spatial and temporal phase, (ii) in temporal phase but spatial antiphase, (iii) in spatial phase but temporal antiphase, or (iv) in spatial and temporal antiphase (for counterphase flicker, this is identical to condition(i)). SSVEPs to monocular and binocular viewing were comparable in magnitude, consistent with \textit{ocularity invariance}. 

An interesting finding from our study is that monocular responses persist in neural signals under binocular viewing conditions. Binocular stimuli presented in temporal anti-phase generated SSVEPs at the fundamental frequency and its odd integer harmonics -- signals that represent the monocular responses to stimuli. Computationally, this finding could only be explained when parallel monocular channels were added to the binocular two-stage contrast gain control model.

We believe this work will be of interest to the Vision Research audience as it advances our understanding of binocular combination and highlights the value of computational modeling in bridging psychophysical and neurophysiological evidence. We hope you find the work to be of sufficient novelty and interest for inclusion in the journal. 


Sincerely,

Bruno Richard
\vspace{-1em}

Daniel H. Baker

\end{document}